% !TEX program = xelatex

\documentclass{resume}
%\usepackage{zh_CN-Adobefonts_external} % Simplified Chinese Support using external fonts (./fonts/zh_CN-Adobe/)
\usepackage{zh_CN-Adobefonts_internal} % Simplified Chinese Support using system fonts

\begin{document}
\pagenumbering{gobble} % suppress displaying page number

\name{Xiang Chen} - {\kaishu 陈 翔}

\basicInfo{
  \email{wasdnsxchen@gmail.com} \textperiodcentered\ 
  \phone{(+86) 177-0594-8606} \textperiodcentered\ 
  \github[]{} https://github.com/Wasdns}

\section{\faGraduationCap\ Education}
\datedsubsection{\textbf{Peking University (PKU)}, Beijing, China}{2020 -- Present}
\textit{Research Assistant} in N2Sys Lab, EECS.
\datedsubsection{\textbf{Institute of Computing Technology, Chinese Academy of Sciences}}{2019 -- Present}
\textit{Visiting Student} in Advanced Computer System Research Center.
\datedsubsection{\textbf{Fuzhou University (FZU)}, Fujian, China}{2019 -- Present}
\textit{Master Student} in Computer Science, expected in March 2022.
\datedsubsection{\textbf{Fuzhou University (FZU)}, Fujian, China}{2015 -- 2019}
\textit{B.S.} in Computer Science.

\section{\faUsers\ Research Projects}
\datedsubsection{\textbf{Data Plane Program Deployment}}{2018 -- Present}
\emph{Optimizing the data plane program deployment in programmable networks.}
\begin{itemize}
  \item \textbf{SPEED (ICNP'20)}: Simultaneously achieving both high resource efficiency and high end-to-end performance in program deployment with the help of an one-big-switch abstraction of substrate network.
  \item \textbf{SRA (GLOBECOM'20)}: Aggregating the resources (e.g., SRAM, switch stages) of multiple programmable switches to satisfy the resource requirements issued by data plane programs.
  \item \textbf{MATReduce (GLOBECOM'18)}: Merging duplicate match operations (e.g., matching source IP address) between match-action tables within a data plane program to reduce per-packet processing latency.
\end{itemize}

\datedsubsection{\textbf{Network Measurement}}{2019 -- Present}
\emph{Bridging the gap between high accuracy and high resource efficiency in network measurement.}
\begin{itemize}
  \item \textbf{MTP (INFOCOM'21)}: Placing measurement tasks (e.g., a count-min sketch) while avoiding control plane (i.e., links and monitoring servers) overload when collecting events (e.g., sketch values) at runtime. 
  \item \textbf{ApproSync (ICNP'20)}: Synchronizing state values (e.g., counter values) directly from switch ASICs to the control plane for low latency while controlling the sending rate of state values to avoid state loss.
\end{itemize}

\datedsubsection{\textbf{Network Function Virtualization (NFV)}}{2018 -- Present}
\emph{Accelerating service function chains (SFCs) with programmable switches and timely scaling decisions.}
\begin{itemize}
  \item \textbf{FastScale (WNA'20)}: Scaling out a whole SFC rather than a single NF to achieve fast convergence. 
  \item \textbf{P4SC (IM'19, IEEE ACCESS)}: Offloading SFCs to programmable switches for high performance.
\end{itemize}

% Reference Test
%\datedsubsection{\textbf{Paper Title\cite{zaharia2012resilient}}}{May. 2015}
%An xxx optimized for xxx\cite{verma2015large}
%\begin{itemize}
%  \item main contribution
%\end{itemize}

\section{\faCogs\ Skills}
\begin{itemize}[parsep=0.5ex]
  \item Academic Services: ICC 2021 Reviewer (NGNI track)
  \item Programming Languages: P4 (2016-present), Python, C, C++
  \item Platforms: Linux (Ubuntu releases), Barefoot Tofino software development environment
  \item Languages: English (CET6), Chinese (native speaker)
  \item Sports: Soccer (defence midfielder, also a big fan of Cristiano Ronaldo)
\end{itemize}

%\section{\faHeartO\ Honors and Awards}
%\datedline{\textit{\nth{1} Prize}, Award on xxx }{Jun. 2013}
%\datedline{Other awards}{2015}

\section{\faInfo\ Miscellaneous}
\begin{itemize}[parsep=0.5ex]
  \item Homepage: https://wasdns.github.io/
  \item Google Scholar: https://scholar.google.com/citations?user=ZMdsjDUAAAAJ\&hl
\end{itemize}

%% Reference
%\newpage
%\bibliographystyle{IEEETran}
%\bibliography{mycite}
\end{document}
