% !TEX TS-program = xelatex
% !TEX encoding = UTF-8 Unicode
% !Mode:: "TeX:UTF-8"

\documentclass{resume}
%\usepackage{zh_CN-Adobefonts_external} % Simplified Chinese Support using external fonts (./fonts/zh_CN-Adobe/)
%\usepackage{zh_CN-Adobefonts_internal} % Simplified Chinese Support using system fonts
%\usepackage{linespacing_fix} % disable extra space before next section

\begin{document}
\pagenumbering{gobble} % suppress displaying page number

\name{Xiang Chen}

\basicInfo{
  \email{wasdnsxchen@gmail.com} \textperiodcentered\ 
  \phone{(+86) 177-0594-8606} \textperiodcentered\ 
  \github[]{} https://github.com/Wasdns}

\section{\faGraduationCap\ Education}
\datedsubsection{\textbf{Peking University (PKU)}, Beijing, China}{2019 -- Present}
\textit{Research Assistant} in N2Sys Lab, EECS. Advisor: Qun Huang.
%\datedsubsection{\textbf{Institute of Computing Technology, Chinese Academy of Sciences}}{2019 -- Present}
%\textit{Visiting Student} in Advanced Computer System Research Center. Advisor: Qun Huang.
%\datedsubsection{\textbf{Fuzhou University (FZU)}, Fujian, China}{2019 -- Present}
%\textit{Master} in Computer Science, expected in March 2022. Advisor: Dong Zhang.
\datedsubsection{\textbf{Fuzhou University (FZU)}, Fujian, China}{2015 -- Present}
\textit{B.S.} (GPA: 3.1/5) (2019) and \textit{M.S.} (expected in 2022) in Computer Science, Advisor: Dong Zhang.

\section{\faUsers\ Selected Research Projects}
\datedsubsection{\textbf{Data Plane Program Deployment}}{2018 -- Present}
%\emph{Optimizing the data plane program deployment in programmable networks.}
\begin{itemize}
  \item \textbf{SPEED (ICNP'20)}: Achieving both high resource efficiency and high end-to-end performance in program deployment with the help of an one-big-switch abstraction of substrate network.
  \item \textbf{SRA (GLOBECOM'20)}: Aggregating the resources (e.g., SRAM and switch stages) of multiple programmable switches to satisfy the resource requirements issued by data plane programs.
  \item \textbf{MATReduce (GLOBECOM'18)}: Merging duplicate match operations (e.g., matching source IP address) between match-action tables within a data plane program to reduce per-packet processing latency.
\end{itemize}

\datedsubsection{\textbf{Network Measurement}}{2019 -- Present}
%\emph{Bridging the gap between high accuracy and high resource efficiency in network measurement.}
\begin{itemize}
  \item \textbf{MTP (INFOCOM'21)}: Placing measurement tasks (e.g., heavy hitter detection based on Count-Min sketch) while avoiding control plane (i.e., links and monitoring servers) overload when collecting events (e.g., heavy hitters) at runtime. This paper is awarded with \textbf{``Best Paper Candidate''}. 
  \item \textbf{NZE-Sketch (NSDI'21)}: Exploiting compressive sensing to boost sketch accuracy and analyze various design choices of sketch algorithms. 
  \item \textbf{ApproSync (ICNP'20)}: Synchronizing state values (i.e., register values) directly between switch ASICs and the control plane for low latency while controlling the sending rate of state values to avoid state loss.
\end{itemize}

\datedsubsection{\textbf{Network Function Virtualization (NFV)}}{2018 -- Present}
%\emph{Accelerating service function chains (SFCs) with programmable switches and timely scaling decisions.}
\begin{itemize}
  \item \textbf{LightNF (IWQoS'21)}: Simplifying the deployment of service function chains on programmable networks with automatic analysis of NF properties (e.g., resource consumption and offloadability). This paper is awarded with \textbf{``Best Paper Award''}. 
%  \item \textbf{FastScale (WNA'20)}: Scaling out a whole SFC rather than a single NF to achieve fast convergence. 
  \item \textbf{P4SC (IM'19)}: Offloading SFCs to programmable switches for high performance. P4SC is a joint work with Ruijie Networks Co., Ltd. It has been commercialized since 2018 (Datasheet RG-F9500-32CQ, F9500 Series P4 Programmable Data Center Switches).
\end{itemize}

% Reference Test
%\datedsubsection{\textbf{Paper Title\cite{zaharia2012resilient}}}{May. 2015}
%An xxx optimized for xxx\cite{verma2015large}
%\begin{itemize}
%  \item main contribution
%\end{itemize}

\section{\faCogs\ Skills}
\begin{itemize}[parsep=0.5ex]
  \item Academic Services: APNet 2019 (volunteer), ICC 2021 (reviewer), APNet 2021 (poster track reviewer)
  \item Programming Languages: P4 (2016-present), Python, C, C++
  \item Platforms: Linux (Ubuntu releases), Barefoot Tofino software development environment
  \item Languages: English (TOEFL 101: R29 L27 S23 W22), Chinese (native speaker)
  \item Cooperation: I work closely with the NGNI Lab (Zhejiang University) led by Prof. Chunming Wu and Dr. Haifeng Zhou. I'm also a research intern of Pengcheng Lab.
  \item Sports: Soccer (defence midfielder, a former varsity player)
\end{itemize}

%\section{\faHeartO\ Honors and Awards}
%\datedline{\textit{\nth{1} Prize}, Award on xxx }{Jun. 2013}
%\datedline{Other awards}{2015}

%\section{\faInfo\ Miscellaneous}
%\begin{itemize}[parsep=0.5ex]
%  \item Homepage: https://wasdns.github.io/
%  \item Google Scholar: https://scholar.google.com/citations?user=ZMdsjDUAAAAJ\&hl
%  \item Cooperation: I work closely with the NGNI Lab (Zhejiang University) led by Prof. Chunming Wu and Dr. Haifeng Zhou. I'm also a research intern of Pengcheng Lab.
%\end{itemize}

\newpage

\section{\faBook\ Selected Publications}
\begin{itemize}[parsep=0.5ex]
%  \item Homepage: https://wasdns.github.io/
  \item \textbf{Xiang Chen}, Qun Huang, Peiqiao Wang, Hongyan Liu, Yuxin Chen, Dong Zhang, Haifeng Zhou, Chunming Wu. MTP: Avoiding Control Plane Overload with Measurement Task Placement. IEEE INFOCOM 2021. (CCF A, AR: 252/1266=19.9\%). Awarded with \textbf{Best Paper Candidate}! (13/252=5.1\%)
  \item \textbf{Xiang Chen}, Qun Huang, Peiqiao Wang, Zili Meng, Hongyan Liu, Yuxin Chen, Dong Zhang, Haifeng Zhou, Boyang Zhou, Chunming Wu. LightNF: Simplifying Network Function Offloading in Programmable Networks. IEEE/ACM IWQoS, 2021. (CCF B, AR: 64/256=25\%). Awarded with \textbf{Best Paper}! (2/64=3.1\%)
  \item Qun Huang, Siyuan Sheng, \textbf{Xiang Chen}, Yungang Bao, Rui Zhang, Yanwei Xu, Gong Zhang. Toward Nearly-Zero-Error Sketching via Compressive Sensing. USENIX NSDI, 2021. (CCF A, AR: 40/255=15.6\%)
  \item \textbf{Xiang Chen}, Qun Huang, Dong Zhang, Haifeng Zhou, Chunming Wu. ApproSync: Approximate State Synchronization for Programmable Networks. IEEE ICNP, 2020. (CCF B, AR: 31/184=16.8\%)
  \item \textbf{Xiang Chen}, Hongyan Liu, Qun Huang, Peiqiao Wang, Dong Zhang, Haifeng Zhou, Chunming Wu. SPEED: Scalable and High-Performance Deployment for Data Plane Programs. IEEE ICNP, 2020. (CCF B, AR: 31/184=16.8\%)
  \item Hongyan Liu, \textbf{Xiang Chen} (co-first author), Qun Huang, Haifeng Zhou, Dong Zhang, Chunming Wu. SRA: Switch Resource Aggregation for Application Offloading in Programmable Networks. IEEE GLOBECOM, 2020. (CCF C)
  \item \textbf{Xiang Chen}, Yuxin Chen, Qun Huang, Haifeng Zhou, Dong Zhang, Chunming Wu, Junchi Xing. FastScale: Fast Scaling Out of Network Functions. IEEE INFOCOM Workshop on Networking Algorithms (WNA), 2020.
%  \item Dong Zhang, \textbf{Xiang Chen}, Qun Huang, Xiaoyan Hong, Chunming Wu, Haifeng Zhou, Yi Yang, Hongyan Liu. P4SC: A High Performance and Flexible Framework for Service Function Chain. IEEE Access, 2019.
  \item \textbf{Xiang Chen}, Dong Zhang, Xiaojun Wang, Kai Zhu, Haifeng Zhou. P4SC: Towards High-Performance Service Function Chain Implementation on the P4-Capable Device. IFIP/IEEE IM 2019. (CCF C)
  \item \textbf{Xiang Chen}, Dong Zhang, Haifeng Zhou. MATReduce: Towards High-Performance P4 Pipeline by Reducing Duplicate Match Operations. IEEE GLOBECOM 2018. (CCF C)
\end{itemize}

%% Reference
%\newpage
%\bibliographystyle{IEEETran}
%\bibliography{mycite}
\end{document}
